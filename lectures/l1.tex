%\documentclass[]{article}
\documentclass[11pt]{paper}
\usepackage[margin=1in]{geometry}
\usepackage[T1]{fontenc}
\usepackage{mathpazo}
\usepackage{amssymb,amsmath}
\usepackage{ifxetex,ifluatex}
\usepackage{fixltx2e} % provides \textsubscript
% use upquote if available, for straight quotes in verbatim environments
\IfFileExists{upquote.sty}{\usepackage{upquote}}{}
\ifnum 0\ifxetex 1\fi\ifluatex 1\fi=0 % if pdftex
  \usepackage[utf8]{inputenc}
\else % if luatex or xelatex
  \ifxetex
    \usepackage{mathspec}
    \usepackage{xltxtra,xunicode}
  \else
    \usepackage{fontspec}
  \fi
  \defaultfontfeatures{Mapping=tex-text,Scale=MatchLowercase}
  \newcommand{\euro}{€}
\fi
% use microtype if available
\IfFileExists{microtype.sty}{\usepackage{microtype}}{}
\usepackage{natbib}
\bibliographystyle{plainnat}
\ifxetex
  \usepackage[setpagesize=false, % page size defined by xetex
              unicode=false, % unicode breaks when used with xetex
              xetex]{hyperref}
\else
  \usepackage[unicode=true]{hyperref}
\fi
\hypersetup{breaklinks=true,
            bookmarks=true,
            pdfauthor={Suresh Venkatasubramanian},
            pdftitle={GIAN Course on Fairness, Accuracy and Transparency in Machine Learning},
            colorlinks=true,
            citecolor=blue,
            urlcolor=blue,
            linkcolor=magenta,
            pdfborder={0 0 0}}
\urlstyle{same}  % don't use monospace font for urls
\setlength{\parindent}{0pt}
\setlength{\parskip}{6pt plus 2pt minus 1pt}
\setlength{\emergencystretch}{3em}  % prevent overfull lines
\setcounter{secnumdepth}{5}

\title{GIAN Course on Fairness, Accuracy and Transparency in Machine Learning}
\subtitle{Lecture 1}
\author{Suresh Venkatasubramanian}
\date{Dec 12, 2016}
\begin{document}
\maketitle

\section{Review of machine learning
concepts}\label{review-of-machine-learning-concepts}

\begin{itemize}
\itemsep1pt\parskip0pt\parsep0pt
\item
  supervised learning
\item
  (binary) classification
\item
  regression
\item
  Standard models for supervised learning

  \begin{itemize}
  \itemsep1pt\parskip0pt\parsep0pt
  \item
    SVMs
  \item
    decision trees
  \item
    Naive Bayes
  \end{itemize}
\item
  unsupervised learning
\item
  clustering + dimensionality reduction
\item
  neural nets as nonlinear DR.
\item
  Mechanics of learning
\item
  training and test
\item
  generalization bounds
\item
  The learning pipeline
\item
  data wrangling, feature extraction, modeling and so on.
\end{itemize}

\section{Automated Decision Making}\label{automated-decision-making}

While it would be hard to find any part of our daily lives that are
\emph{not} touched by machine learning, the issues of fairness and
accountability really come into play when we think about algorithms that
make \emph{decisions} (or assist with them).

So if we think of the classic binary classification problem where the
goal is to determine a mapping $f $ from the input $X$ to the set
$\{-1, 1\}$, we should instead think of a mapping into the
\emph{decision space} $\{Y, N\}$. More generally, the mapping might not
actually return a binary variable. Rather, it might return some kind of
``risk score'' that we can interpret as a probability of $Y$. Or it
might return a number that we will later threshold: for example the FICO
score is a number whose maximum value is 850: a score of 700 or higher
is deemed quite good.

This decision could be anything: it could be the decision to place an ad
on a web page, or the decision to make a recommendation to an user from
a shopping website, or a recommendation for a route to take when asked
for directions. It could even be the decision about \emph{what} to
return in a search query,

Decisions can be more complex: for example, the decision of how to rank
results on a search query doesn't quite fit the model above. It can be
modeled approximately by assuming that the decision task is to assign a
rank to a search result. But for the purpose of discussion we'll stick
to the binary output model (or in some cases its regression cousin).

What I'd like to do is explore a few cases in some detail to understand
\emph{how} the automated decision-making pipeline works. In some cases,
the very fact that algorithms are being used might be considered
surprising (or shocking!). But for any examination of the benefits (and
dangers) of using automated methods to be thorough, we should walk
through the different domains.

These examples are (unfortunately) very specific to the US context: both
in terms of what's currently going on and in terms of what the law
allows. But I'd encourage you all to think about how these might (or do)
apply in the context of Indian law as well.

\subsection{Criminal Justice}\label{criminal-justice}

Let's start with the most consequential example of the use of automated
methods: in the criminal justice system. There is an entire pipeline of
decisions that start with the question of how the police target areas
(and individuals) for investigation and end with decisions about whether
to release someone from jail on parole or not. This are usually
described by two processes:

\begin{enumerate}
\def\labelenumi{\arabic{enumi}.}
\itemsep1pt\parskip0pt\parsep0pt
\item
  predictive policing
\item
  risk assessment
\end{enumerate}

\subsubsection{Predictive policing}\label{predictive-policing}

You've probably seen the scifi movie Minority Report which describes a
near future where telepathically enhanced individuals ``predict'' crime
before it happens. While predictive policing doesn't claim to go this
far, it is an attempt to use prior crime history to predict crime in two
ways:

\begin{itemize}
\itemsep1pt\parskip0pt\parsep0pt
\item
  \textbf{where} crime is likely to happen
\item
  \textbf{who} might be committing a crime.
\end{itemize}

For the rest of this, we will discuss only the first kind of predictive
model. It is being used widely across police departments. The second
model is of uncertain legality and appears to be the domain of national
intelligence services for the most part.

The goal of spatial predictive policing is to determine where crime
might be more likely to occur. The point of doing this is to manage the
scarce resources of a police department: patrol officers, cars, and so
on. If a department ``knows'' that a certain region of a precinct will
be a crime ``hotspot'' that day, they can provision more officers to
that area while patrolling less risky areas less frequently.

For this purpose, a region of space is often subdivided into
\emph{cells} --- say a few blocks across. The predictive task here is
then: over a fixed time period $t$ (say a day), for each cell $c$ of a
collection of cells, assign a score $s(c, t)$ that is higher if $c$ is
likely to experience more crime during period $t$. In its simplest form,
the score $s(c, t)$ could merely be a binary-valued variable indicating
crime ($1$) or no-crime ($0$).

In order to build this predictor, a learning algorithm might use a
variety of data sources and features: these might include

\begin{itemize}
\itemsep1pt\parskip0pt\parsep0pt
\item
  crime reports to the police in $c$ over ``similar'' time periods in
  the past.
\item
  actual arrests made in $c$
\item
  calls to emergency services (that may or may not result in an arrest
  or even a crime report)
\item
  baseline demographic information in the cell.
\item
  information about special events that might influence crime rates and
  reporting (for example, rioting in the streets after a sports event).
\end{itemize}

\subsubsection{Training data}\label{training-data}

In predictive policing, training data is typically historical records of
crime in the area, possibly qualified by time of year/season (for
example data from the summer is used to predict crime in the summer, and
so on).

\subsubsection{Risk assessment}\label{risk-assessment}

Risk assessment addresses the ``other'' side of criminal justice, after
someone has been arrested for a crime. Risk assessment tools come into
play at different stages of the conviction, sentencing and release
pipeline, for example:

\begin{itemize}
\itemsep1pt\parskip0pt\parsep0pt
\item
  should an undertrial be released on bail prior to the hearing
\item
  what duration sentence should a convicted individual be given within
  the guidelines laid out by the law.
\item
  should an inmate be granted parole or sent back to prison.
\end{itemize}

Notice that risk assessment, unlike the standard use of predictive
policing, is individual-based. In each case, the ``risk'' being assessed
here is whether the individual represents an unreasonable risk (of
fleeing before trial, of receiving an undeservedly lenient sentence, of
committing new crimes) if released into the general population. The risk
assessment algorithm will take as input features associated with an
individual $p$ and output a \emph{risk score} $r(p)$ between $0$ and $1$
(or in some interval) where $1$ represents a high risk of danger if $p$
is released and $0$ represents no risk.

Since these decisions are made for an individual, the features used tend
to be more specific, they might include:

\begin{itemize}
\itemsep1pt\parskip0pt\parsep0pt
\item
  demographics information about the individual
\item
  past history and nature of crimes
\item
  social milieu and network.
\item
  general psychological and emotional state.
\end{itemize}

For example, when an inmate comes up for parole, they are put through a
battery of questions (137 in some cases) that span the range of topics
mentioned above, and these are fed into a predictor to output a risk
level between $1$ and $10$ of whether they'll recommit a crime within 6
months of release (or within 2 years). \citep{angwin_machine_2016}

The scores produced by risk assessment tools are typically not
determinative: they are merely one feature provided to a judge to do
with as they see fit. In fact, there are studies suggesting that the
degree to which judges take risk assessments into consideration vary
widely and inconsistently.

\subsubsection{Training data}\label{training-data-1}

In risk assessment, training data comes from prior records of bail
violations, recidivism and so on. General demographics factors might
also be used to condition the data (such as national or state averages
of recidivism and so on)

\subsection{Hiring}\label{hiring}

For reasons of both scale (the number of applicants applying for jobs)
and diversity (trying to ensure a broad representation of skills in the
workforce), many companies are trying to go beyond the standard
human-interview based model for hiring to take advantage of data
analytics. The general goal is to predict which potential employees are
likely to ``do well'' at their job. This can be interpreted in a variety
of ways:

\begin{itemize}
\itemsep1pt\parskip0pt\parsep0pt
\item
  likely to stay at the job for a sufficient period of time
  (\emph{retention})
\item
  likely to have high productivity in the job (\emph{productivity})
\item
  likely to be content at the job (\emph{satisfaction}).
\end{itemize}

Note that the criteria don't necessarily measure the same attributes, an
employee might be willing to stay at their job without being productive,
and might be productive and yet unsatisfied.

The idea therefore is to use predictive tools as well as new techniques
for processing applicants to make such an assessment. Formally, just as
above, the predictive tool would assign a score $s(p)$ to an applicant
$p$ and then present this score to the hiring manager as part of a
dashboard of information.

What makes hiring interesting is the innovation that has taken place in
\emph{how} the interviews are conducted and analyzed. It is now common
for an applicant to undergo a fully automated initial screening or full
interview, without any human intervention. These interviews can be video
or text-based. In either case, the applicant is sent to a website to
login to at a time of their choosing, and then the interview proceeds.

\begin{itemize}
\itemsep1pt\parskip0pt\parsep0pt
\item
  in a text-based interview, the applicant sees questions on the screen
  and has to type out their answers. The answers are typically timed,
  and while no attempt is made to prevent applicants from ``looking up''
  answers, the questions are sufficiently open-ended as to preclude this
  within the time specified. Such a process might even occur for tech
  interviews where an applicant might be asked to write code or pseudo
  code.
\item
  In a video-based interview, the applicant might either see questions
  from a pre-recorded video, or might again see text questions. In
  either case, the applicant records a video response (usually via their
  webcam), and this is sent off for further analysis.
\end{itemize}

Once the responses have been received, the predictive model goes to
work. It extracts features from the transcript: these might be text
features - words, phrases, and so on. These might also be cues from the
video. Face recognition software is used to identify micro-expressions
in the face while the interviewee responds, and these are tagged with
``standard'' interpretations (such as are used for example in
lie-detection systems).

This entire list of features is assembled into the input to a predictor,
which then returns the desired score.

\subsubsection{Training data}\label{training-data-2}

When a third party vendor works with a company to build a predictive
model, they might build a custom model for that organization. Every
company might have different hiring needs --- a tech startup might be
looking for different characteristics than a multinational hotel chain
for example. A common way to acquire training data is therefore to look
at past employee records to see how well they did, and build a
predictive model from it.

\subsection{Loans}\label{loans}

When a bank issues a loan, their primary concern is credit-worthiness:
will the loanee pay back the loan promptly and in full? For centuries,
lenders have used many different strategies to assess credit-worthiness,
including factors like collateral (what items can be used as a risk
mitigator in case of default), history of loan repayment, behavioral
attributes (does this individual indulge in risky behavior) and even
background cultural context.

In the US, loan granting was an area that was rife with racism and other
forms of bias: indeed the term \emph{redlining} comes from the racist
mortgage lending practices of the 1930s. In the 60s, an attempt to
reform this process led to the development of what is now known as the
FICO score: it ranges from 0-850 with a higher score indicating a more
trustworthy borrower.

Till the late 90s, the FICO score was the de facto tool for assessing
credit-worthiness. It was used to both grant loans and even set the rate
of a loan. But with the wild fluctuations in the stock market in the
2000s (the tech market crash as well as the housing crisis), the FICO
score is turning out to be less accurate as a predictor (because even
potentially reliable borrowers have weak credit as seen through the FICO
lens) and there is now a growing market in predictive models for
credit-worthiness that take advantage of all the data out there to be
mined.

To formalize the predictive scenario as before, we assume that the model
is presented with an individual $p$ and the goal is to output a score
$s(p)$ that will indicate credit-worthiness. The lender might then use
the score $s(p)$ and other information to generate an interest rate
$r(p)$ for the individual, or a denial.

Features that are now often used (but not always) to determine whether
and how much for a loan should be granted include:

\begin{itemize}
\itemsep1pt\parskip0pt\parsep0pt
\item
  social media presence: a facebook feed that indicates risk taking
  behaviour might reduce the score for an individual.
\item
  online reviews: if you're a small business and you're on Yelp, your
  reviews might be taken into account when decided how valuable your
  business is.
\end{itemize}

What is perhaps more interesting is that there are now proposals to use
extended social networks to model an individual's credit worthiness. For
example, the Chinese government is planning to compute a ``citizen
score'' that will be used for a number of government services including
loans and jobs. Their proposal is to use not just individual information
but information about one's circle of friends (as captured by social
media) to determine one's ``value'', with the full intention to use (for
example) statements critical of the government --- even by one's friends
--- as part of this computation.

\subsubsection{Training data}\label{training-data-3}

Training data for loan granting can come from a bank's prior customers.
It is less clear how to build models for training from social media: one
possibility is that the model builder tracks the social media presence
of successful borrowers, and also obtains (via data brokers) information
about rejected borrowers. There is a flourishing market in third party
data scraped from various sources, involving large brokers like Acxiom
and others.

\section{Definitions of Fairness}\label{definitions-of-fairness}

At the heart of discussions of bias is the corresponding question of
fairness and nondiscrimination. What does it mean for an algorithm to be
``fair'' or ``unfair''? It turns out that this question will not lend
itself to an easy answer, and will force us to return to the underlying
philosophical and ethical questions around the idea of a just society.
But in the meantime we can review the different definitions of fairness
that different researchers have either proposed or have modeled based on
inspiration from the real world.

In this discussion, we will only talk about the definitions themselves.
We will defer discussions of mitigation (how to make sure biased
patterns are not found) to a later lecture.

\subsection{Bias based on learning
rules}\label{bias-based-on-learning-rules}

The first notion of bias in data mining is due to Pedreschi, Ruggieri
and Turini \citep{pedreshi_discrimination-aware_2008}, and originates
from the area of itemset mining. In this setting, the data is
represented as a relation with attributes $a_i$. A transaction can be
thought of as a row in this table. An \emph{itemset} is a collection of
attribute settings: for example, ``pin code - 560 076'', ``state = KN'',
and so on. The support of an itemset $X$ is the fraction of transactions
that satisfy it.

Consider a classification rule $X \implies Y$ learned by a data mining
system on a table. We can quantify our \emph{confidence} in this rule by
measuring a proxy for the empirical probability $\Pr(Y \mid X)$:
specifically,

\[ \text{conf}(X \implies Y) = \frac{\text{support(X,Y)}}{\text{support(X)}} \]

\subsubsection{Encoding beliefs about
bias}\label{encoding-beliefs-about-bias}

As we shall see, all attempts to determine bias in ML must have some
external statement about what kinds of decisions might be considered
biased: this is not something the modeling process will discover by
itself. In the case of association rule mining, this information is
encoded by marking certain itemsets $X$ as \emph{potentially
discriminatory} (PD). For example, any rule of the form ``If female, do
something'' has the itemset ``gender = Female'' as an antecedent and
might be discriminatory in certain contexts. So we will assume \emph{a
priori} that a set of such PD itemsets is provided to us.

How do we measure bias then? The idea is to determine whether the PD
itemsets made a \emph{material difference} in the confidence of the
learned association rule. Specifically, let the \emph{extended lift} of
a rule $A, B \implies C$ with respect to $B$ be the quantity

\[ \text{elift} = \frac{\text{conf}(A, B, \implies C)}{\text{conf}(A \implies C)}\]

Then they define a PD itemset $B$ as $\alpha$-discriminatory if
$\text{elift} \ge \alpha$.

Note that if the attribute $C$ is binary, then the implications
$A \implies C$ and $A \implies !C$ are related, and a stronger notion of
the extended lift can be defined in that setting.

\subsection{Bias based on statistical
discrepancy}\label{bias-based-on-statistical-discrepancy}

At its core, the definition of bias is based on the idea of influence:
how much does a particular variable appear to affect a decision? The
next definition is more explicit, focusing directly on the outcome.

Assume again that we have a data set with attributes $a_i$. Suppose the
goal of the decision-making process is to determine some outcome encoded
as $C$ where $C =1$ is the positive outcome. We will encode beliefs
about bias by specifying a particular itemset of the form $a_i = v$. For
example ``gender = FEMALE'' or ``race = non-white''. The key to this
notion of bias measurement is to look at the (conditional) difference in
the favorable outcome for the selected group (as encoded by the
itemset). The conditional probability of the good outcome for the
subgroup is given by

\[ \frac{\Pr( C = 1 \mid X)}{\Pr(X)} \]

Similarly, the conditional probability for the rest can be written as

\[ \frac{\Pr( C = 1 \mid \overline{X})}{\Pr(\overline{X})} \]

Then one measure of statistical bias is the difference between these
numbers. \citep{kamiran_data_2012}

\subsubsection{Disparate Impact}\label{disparate-impact}

The above measure of bias is used in the United Kingdom to quantify
gender-based discrimination. In the US, we instead take the \emph{ratio}
of the two numbers. This is the basis of the ``80\% rule'' that says
that the ratio of the two conditional probabilities must not be less
than 4/5. \citep{feldman_certifying_2015}

In both cases, the discriminatory attribute is fixed ahead of time. In
the US for example, attributes like race, gender, ethnicity, and age are
subject to such scrutiny. For age (which is a continuous variable), the
binary variant is whether the age is less than or more than 40.

\subsubsection{Equalizing odds}\label{equalizing-odds}

One argument against bias measurements based on statistical discrepancy
is that it may be that in the population at large, there is a natural
and ``unbiased'' difference in skills based on the protected attribute
$X$. For example, if the job involves lifting of heavy objects, then an
employer might argue that the average male is better suited to the task
than the average female by virtue of having better upper body strength,
and so this discrepancy will naturally appear in the hiring ratios
without being discriminatory.

One way to address this issue is to require not that the different
groups be treated equally, but that the decision procedure is equally
accurate (or inaccurate) on the two groups. Thus far we have assumed a
decision variable $C$ and the group attribute $X$. Let us now introduce
the ground truth variable $C^*$. In other words, if $p$ is an
individual, then $C(p)$ is the decision made by an algorithm and
$C^*(p)$ is the decision it should have made.

The false positive rate for the predictor is
$FP = \Pr(C = 1 \mid C^* = 0)$. Similarly the true positive rate is
$TP = \Pr(C = 1 \mid C^* = 1)$. Suppose we evaluate $FP$ and $TP$
conditioned by group membership. Then we want that these numbers are the
same for the two groups.

Another way to address the issue of equalizing odds is to instead try to
equalize the probability of misclassification conditioned by group. In
other words,

\[ \Pr(C \ne C^* \mid X = 1) = \Pr(C \ne C^* \mid X = 0)\]

\citep{zafar_fairness_2016}

\subsection{Fairness through
Awareness}\label{fairness-through-awareness}

In all of the above, we're really considering the problem of
nondiscrimination: how to make sure that different groups are treated
fairly, for differing definitions of fair. But we can also take a very
different perspective on the problem of fairness. Rather than ensuring
that groups are treated similarly, we attempt to ensure that
\emph{individuals} are treated fairly. A formal notion of individual
fairness was given by Dwork et. al. in 2011 \citep{dwork_fairness_2012}

Assume that for a given task, we have some sense of how close
individuals are with respect to competency for that task For example
this could be encoded as some kind of distance function $d(p, p')$.
Further, assume that the outcome is the result of a randomized
procedure, so that over repeated trials of the algorithm we can talk
about the probability of the outcome $C = 1$ and more generally
associate a rule with a distribution over outcomes for each individual.
Then the idea of \emph{individual fairness} says that two individuals
who are close should be treated similarly. Formally, suppose we have
some measure of distance between distributions $D$. Then

\[ d(x,y) \le \epsilon \implies D(C(x), C(y)) \le \delta\].

Notice that this is a fundamentally different notion from group
fairness. Indeed it is easy to design scenarios where one of these
notions is satisfied but not the other, and so on.

\subsection{Fairness in Sequential
learning}\label{fairness-in-sequential-learning}

Thus far, we've considered fairness in the supervised (and batch) model
of learning. There's a classifier, training data, test data and so on.
But automated decision-making is an ongoing process, where the model
might keep adapting to the data and making new decisions. While there's
a whole issue of consistency in decision-making that we will not
address, there's a more basic question of what fairness means in such a
context.

One of the standard settings for thinking about sequential learning is
the bandit model. While I won't go into details about the bandit
framework in learning, I'll set up some basic notation.

Assume the bandit has $k$ arms. At each step $t$ the algorithm chooses
one arm $i_t$ from a distribution $\pi^t$ over $[k]$ and receives the
(stochastic) reward drawn $r_j$ drawn from an \textbf{unknown}
distribution $\mathcal{D}_j$ with mean $\mu_j$. It then updates $\pi^t$
to get $\pi^{t+1}$ and the process repeats. The \emph{regret} of the
algorithm is the difference between its expected reward and the best
expected reward. Since the best expected reward is obtained by always
choosing the arm $i^*$ with the largest expected reward $\mu_{i^*}$, we
can write the regret as

\[ T \cdot \mu_{i^*} - E \sum \mu_{i^t} \]

Let us denote the \emph{history} of the process $h^t$ as the sequence of
decisions made by the algorithm till time $t$. This includes all choices
of arms to pull and all rewards obtained. At time $t$, we can now define
the quantity $\pi_{j \mid h^t}$ as the probability that the algorithm
chooses arm $j$ at time $t$ given the entire history $h^t$.

We're now ready to define a notion of fairness for sequential learning.
Intuitively, the definition attempts to capture the idea that if the
baseline value of one arm is more than another, the algorithm should
choose it. Formally, an algorithm is $\delta$-fair\citep{joseph_fairness_2016} if for all
distributions $\mathcal{D}_1, \ldots, \mathcal{D}_k$, with probability
at least $1-\delta$ over the history, for all $t$ and all arms
$j, j' \in [k]$,

\[ \pi^t_{j\mid h} > \pi^t_{j' \mid h} \text{only if  } \mu_j > \mu_{j'} ​\]

The intuition behind this definition is this: the only justification for
choosing one arm over another should be if there's a measurable benefit
to taking that arm. If we (for example) think of arms as individuals,
then we are trying to capture the idea a la individual fairness that the
only justifiable reason for different treatment is different underlying
quality (this is the contrapositive of the way individual fairness is
framed).

% \subsection{Notes}\label{notes}

% CCTNS is a system being used in India for predictive policing

% CIBIL is the Indian equivalent of FICO

% the IFMR is an institute in Chennai that does research into loans and
% has found that caste appears to play a significant role in the process.

\bibliography{fatcs}
\end{document}
